\newpage
\section{Introducción teórica}
Los polinomios de Chebychev son funciones ortogonales utilizadas en el diseño de filtros eléctricos por su capacidad para controlar la respuesta en frecuencia. Los filtros Chebychev ofrecen una alta selectividad al permitir pendientes más pronunciadas entre la banda de paso y la banda de parada, a costa de ondulaciones controladas en la respuesta. Son comunes en aplicaciones de procesamiento de señales y comunicaciones. Algunas de sus ventajas y características son:

\begin{itemize}
    \item \textbf{Mayor eficiencia en diseño:} Para un mismo nivel de atenuación, requieren un orden más bajo que otros filtros como Butterworth, reduciendo costos y complejidad. Ofrecen un cambio rápido entre las bandas, mejorando la capacidad de rechazo de señales fuera de la banda deseada. A su vez se pueden adaptar para diferentes aplicaciones, desde sistemas de audio hasta comunicaciones inalámbricas.
	\item \textbf{Características de atenuación específicas:} Los polinomios de Chebychev permiten diseñar filtros con características de atenuación específicas en la banda de paso. Esto significa que se pueden lograr atenuaciones más pronunciadas en la banda de paso en comparación con otros polinomios, como los de Butterworth.

    \item \textbf{Tipos de filtros:} Los filtros Chebychev se destacan por su respuesta en frecuencia única, que varía dependiendo del tipo (I o II) y del diseño específico. A continuación, se detalla cómo se comportan en ambos casos:
    
    \textbf{Filtro Chebychev Tipo I:}  Exhiben ondulaciones en la banda de paso, cuya magnitud está controlada por el parámetro de ripple ().La banda de parada es completamente monótona, es decir, no presenta ondulaciones.
    
    \textbf{Filtro Chebychev Tipo II:} La banda de paso es completamente plana (sin ondulaciones), lo que es ventajoso en aplicaciones que requieren estabilidad de amplitud dentro de esta región. Presentan ondulaciones en la banda de parada, lo que permite un diseño más eficiente en términos de orden.

	
\end{itemize}
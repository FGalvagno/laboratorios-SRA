\section{Conclusión}
En este trabajo de laboratorio, se lograron estudiar cuatro circuitos y observar sus limitaciones al momento de armarlos físicamente.

El primer circuito, amplificador diferencial, solamente permite amplificar con una ganancia diferencial de un poco más de 3 veces en lugar de 4 veces como lo esperábamos según el análisis teórico y la simulación. El segundo circuito se comportó muy similar al caso estudiado en la simulación y en el análisis teórico del mismo. El tercer circuito se asemejaba al simulado y se comportaba según el análisis teórico planteado para valores de tensión de entrada bajos. Al aumentar demasiado la tensión de entrada, el circuito se vio limitado físicamente por el valor de su alimentación y pasó a saturación. Además, se logró verificar tanto por simulaciones como por mediciones en el laboratorio los efectos que traen la no linealidad de los componentes. El último circuito armado funcionó de manera correcta al armarlo físicamente hasta que se trabajó con alimentación asimétrica; a partir de este momento se encontraron diferencias notables entre el modelo simulado y lo medido.

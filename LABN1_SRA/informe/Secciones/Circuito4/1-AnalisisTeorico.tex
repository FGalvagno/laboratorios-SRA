\section{Circuito N° 4: Comparador con Histéresis (Schmitt Trigger Inversor)}



\subsection{Análisis Teórico}

Se propone un circuito Schmitt Trigger inversor implementando el amplificador operacional LM324. Se realiza el análisis teórico del circuito que se muestra en la siguiente figura. 


\begin{figure}[h!]
    \centering
    \includegraphics[width=0.50\linewidth]{Secciones/Circuito4/Circuito 4.png}
    \caption{comparador con histéresis}
    \label{fig:Circuito4}
\end{figure}

Los correspondientes valores tensiones de alimentación, referencia y valores de los componentes son los siguientes:

\[V+ = 10V\]
\[V- = 0V\]
\[V_{ref} = 2V\]
\[R_1 = R_2 = R_4 = 10K\Omega\]
\[R_3 = 2K\Omega\]

Se analizan los umbrales de conmutación del circuito. Se consideran las realimentaciones positivas y negativas.

\[v^- = k_1 * V_i\]
\[v^+ = k_2 * (V_o - V_{ref}) + V_{ref}\]
Con:
\[k_1 = \frac{R_2}{R_1 + R_2}\]
\[ k_2 = \frac{R_3}{R_3 + R_4}\]
\subsubsection{Condición \texorpdfstring{$v_d < 0$}{vd < 0}}
Siendo vd = v+ - v-. Si la tensión vd < 0, se considera que en este caso la tensión de salida tiene un valor inicial Vo=Vcc y luego pasa a tener un valor Vo=Vee, debido a la acción de conmutación.  Entonces se tiene:

\[v_d = v^+ - v^- < 0 \Rightarrow V_o = V_{ee}\]
\[v^+ < v^- \]
\[k_2 * (V_o - V_{ref}) + V_{ref} < k_1 * V_i\]
\[\frac{k_2}{k_1} * (V_o - V_{ref}) + \frac{V_{ref}}{k_1} < V_i \]
\[\frac{k_2}{k_1} * V_o + \frac{1 - k_2}{k_1} * V_{ref} < V_i \]
Se tiene:
\[ k_1 = \frac{R_2}{R_1 + R_2} = \frac{10K\Omega}{10K\Omega + 10K\Omega} = 0.5\]

\[ k_2 = \frac{R_3}{R_3 + R_4} = \frac{2K\Omega}{2K\Omega + 10K\Omega} = 0.166\]

Reemplazando en la expresión anterior:

\[\frac{0.166}{0.5} * 10V + \frac{1 - 0.166}{0.5} * 2V < V_i \]
\[3.32V + 3.34V < V_i \]
\[V_i > 6.66V \Rightarrow V_o = V_{cc}\]
Para nuestro caso Vcc= 10V.

\[V_i > 6.66V \Rightarrow V_o = 10V\]

\subsubsection{Condición \texorpdfstring{$v_d > 0$}{vd > 0}}
Se tiene que vd > 0. Previo a esta condición, el circuito tendrá un valor de salida de Vo=Vee, por lo que conmutará de un valor de Vo=Vee a Vo= Vcc. Se desarrolla la condición planteada:
\[v_d = v^+ - v^- > 0 \Rightarrow V_o = V_{cc}\]
\[v^+ > v^-\]

\[ k_2 * (V_o - V_{ref}) + V_{ref} > k_1 * V_i\]

\[\frac{k_2}{k_1} * V_o + \frac{1 - k_2}{k_1} * V_{ref} > V_i \]
Como Vee= 0v, se tiene:
\[\frac{0.166}{0.5} * 0V + \frac{1 - 0.166}{0.5} * 2V > V_i \]
\[\frac{1 - 0.166}{0.5} * 2V > V_i \]
\[V_i < 3.336V \Rightarrow V_o = V_{ee}\]
Para nuestro caso:
\[V_i < 3.336V \Rightarrow V_o = 0V\]

\subsection{Simulación}

\subsubsection{Simulación con alimentación simétrica (Vcc= 10v y Vee= -10v)}

Se realiza la simulación del circuito con alimentación simetrica en el software LTSpice.

\begin{figure}[H]
    \centering
    \includegraphics[width=0.8\linewidth]{Circuito4.png}
    \caption{Circuito con alimentación simétrica}
    \label{fig:enter-label}
\end{figure}
Se inyecta una señal senoidal de entrada con las siguientes características:

\[ \widehat{V_i} = 10 \, \text{V} \, (\text{Amplitud pico}) \]
\[frecuency= 1KHz\]

En la siguiente figura se muestra la señal de salida Vo (color verde)  y la señal de entrada Vi (color azul) con los siguientes valores medidos:
\begin{figure}[H]
    \centering
    \includegraphics[width=1.0\linewidth]{Secciones/Circuito4/Circuito 4 - Simulación con alimentación simétrica.png}
    \caption{Señal de salida de circuito con alimentación simétrica}
    \label{fig:SimulacionConAlimentacionSimetrica}
\end{figure}

Se observa en la siguiente figura la señal de entrada en el pin v+ (en color azul).

\begin{figure}[H]
    \centering
    \includegraphics[width=1.0\linewidth]{Secciones/Circuito4/Circuito 4 - Vi no atenuada.png}
    \caption{Señal de entrada en v- sin distorsión}
    \label{fig:ViNoAtenuada}
\end{figure}
- Umbral de conmutación de valor Vo=Vee a Vo=Vcc. 
\[V_{i_{\text{sim}}}= 2.06V\]
Lo cual en teoría se verifica con la condición teórica analizada anteriormente:
\[V_{i_{\text{teórica}}} < 3.336V \Rightarrow \text{con} \; V_{o_{\text{inicial}}} = V_{ee}\]

- Umbral de conmutación de valor Vo= Vcc a Vo= Vee 
\[V_{i_{\text{sim}}}= 6,31V\]
El valor de tensión de umbral de simulación se aproxima al valor teórico de conmutación.
\[V_{i_{\text{teórica}}} > 6.66V \Rightarrow V_{o_{\text{inicial}}}= V_{cc} \]
Se concluye el comportamiento del circuito Schmitt Trigger Inversor, con los siguientes umbrales:

\begin{equation}
    \text{Valores \space de \space conmutación \space =}
    \begin{cases}
      V_{ee} \rightarrow V_{cc}, &  Vi\leq-2,06V \\
      V_{cc} \rightarrow V_{ee}, &  Vi\geq6,31V
    \end{cases}
  \end{equation}

Se tiene que los valores reales de la salida difieren de los valores teóricos Vo=Vcc= 10V y Vo=Vee= -10V. Se obtuvieron los siguientes valores de salida Vomáx= 8,53V y Vomín=-9,2V.  Esto es debido a que el amplificador operacional implementado no tiene la característica Rail-to-Rail. 


\subsubsection{Simulación con alimentación asimétrica (Vcc=10V y Vee=0v)}

Se realiza la simulación del circuito con alimentación asimetrica en el software LTSpice.

\begin{figure}[h!]
    \centering
    \includegraphics[width=0.75\linewidth]{Secciones/Circuito4/Circuito 4 - Alimentación asimétrica.png}
    \caption{Circuito N° 4 - Alimentación asimétrica}
    \label{fig:AlimentacionAsimetrica}
\end{figure}

Se inyecta una señal senoidal de entrada con las siguientes características:

\[{\widehat{V}}_i= 10v \space (Amplitud \space pico)\]
\[frecuency= 1KHz\]

 En la siguiente figura se muestra la señal de salida Vo (color verde)  y la señal de entrada Vi (color azul).
 
\begin{figure}[H]
    \centering
    \includegraphics[width=1.0\linewidth]{Secciones/Circuito4/Circuito 4 - Simulación con alimentación asimétrica.png}
    \caption{Señal de salida de circuito con alimentación asimétrica}
    \label{fig:SimulacionConAlimentacionAsimetrica}
\end{figure}
\begin{figure}[h!]
    \centering
    \includegraphics[width=1.0\linewidth]{Secciones/Circuito4/Circuito 4 - Vi atenuada.png}
    \caption{Distorsión de señal de entrada en v-}
    \label{fig:ViAtenuada}
\end{figure}
Se observa en la medición que la señal en el pin v- del amplificador (señal de entrada atenuada) que la misma se distorsiona en el semiciclo negativo. Esto se debe a que los transistores internos que conforman el amplificador operacional no se encuentran polarizados para procesar una señal sinusoidal simétrica.  Esta distorsión no ocurre para el circuito con alimentación simétrica Vcc=10V y Vee=-10V.

Debido a la característica anteriormente nombrada del operacional implementado, se tiene que los valores máximos y mínimos de salida de la simulación difieren de los valores teóricos.

Se obtienen los siguientes valores de tensión de umbral:

- Umbral de conmutación de valor Vo= v- a Vo= v+ 
\[V_{i_{\text{sim}}}= 3,07V\]
Lo cual en teoría se verifica con la condición teórica analizada anteriormente:
\[V_{i_{\text{teórica}}} < 3.336V \Rightarrow con \space V_{o_{\text{inicial}}} = V_{ee}\]
- Umbral de conmutación de valor Vo= v+ a Vo= v- 
\[V_{i_{\text{sim}}}= 6,31V\]

El valor de tensión de umbral de simulación se aproxima al valor teórico de conmutación.
\[V_{i_{\text{teórica}}} > 6.66V \Rightarrow V_{o_{\text{inicial}}}= V_{cc} \]
Se concluye el comportamiento del circuito Schmitt Trigger Inversor propuesto, con los siguientes umbrales para una alimentación asimétrica es la siguiente:

\begin{equation}
    \text{Valores \space de \space conmutación \space =}
    \begin{cases}
      V_{ee} \rightarrow V_{cc}, &  Vi\leq3,07V \\
      V_{cc} \rightarrow V_{ee}, &  Vi\geq6,31V
    \end{cases}
  \end{equation}
  
Se tiene que los valores reales de la salida difieren de los valores teóricos Vo=Vcc= 10V y Vo=Vee= -10V. Se obtuvieron los siguientes valores de salida Vomáx= 8,53V y Vomín=-700mV.  Esto es debido a que el amplificador operacional implementado no tiene la característica Rail-to-Rail. 

\subsection{Implementación}

Se realizaron mediciones a partir del PCB confeccionado para el trabajo práctico.

\begin{figure}[H]
    \centering
    \includegraphics[width=0.5\linewidth]{Secciones//Circuito4/pcb4.png}
        \caption{Salida del comparador Vo\_D y sus entradas Vi\_D y Vref\_D en el PCB}
    \label{fig:enter-label}
\end{figure}

\subsubsection{Medición con alimentación simétrica (Vcc= 10v y Vee= -10v)}
Para las mediciones con alimentacion simétrica, se observa que la salida coincide con el modelo simulado bastante bien, incluso alcanzando niveles de conmutación similares.

\begin{figure}[H]
    \centering
    \includegraphics[width=0.75\linewidth]{Secciones/Circuito4/SDS00001.jpg}
    \caption{Señal de salida del circuito con alimentación simétrica. En amarillo la señal de entrada y en violeta la señal de salida del comparador. Vref = 1V}
    \label{fig:enter-label}
\end{figure}

Los valores que se miden con los cursores para la conmutación de la salida son:

\begin{equation}
    \text{Valores \space de \space conmutación \space =}
    \begin{cases}
      V_{ee} \rightarrow V_{cc}, &  Vi\leq-2,40V \\
      V_{cc} \rightarrow V_{ee}, &  Vi\geq6,90V
    \end{cases}
  \end{equation}

\begin{figure}[H]
    \centering
    \includegraphics[width=0.75\linewidth]{Secciones/Circuito4/SDS00002.jpg}
    \caption{Niveles de conmutación de la señal de salida.}
    \label{fig:enter-label}
\end{figure}

\subsubsection{Medición con alimentación asimétrica (Vcc= 10v y Vee= 0v)}
Para la alimentación asimétrica los niveles de conmutación de la señal no eran del todo consistentes. Como se mencionó anteriormente, al no estar polarizados en tensión negativa los transistores internos del integrado, se produce una distorsión en el semiciclo negativo de la onda. En este caso, la distorsión se traduce en un recorte completo del semiciclo positivo de la señal, diferenciándose completamente de la simulación realizada.

\begin{figure}[H]
    \centering
    \includegraphics[width=0.75\linewidth]{Secciones/Circuito4/SDS00010.jpg}
    \caption{Niveles de conmutación de la señal de salida para alimentación asimétrica. En amarillo, la señal medida en v-.}
    \label{fig:enter-label}
\end{figure}

\subsection{Variación del Ciclo de Histéresis con la Tensión de Referencia}

Se observa como cambiaría el lazo de histéresis para diferentes valores de Vref.

\begin{figure}[H]
    \centering
    \includegraphics[width=0.75\linewidth]{Secciones/Circuito4/Circuito 4 - Vref1.png}
    \caption{Ciclo de Histéresis con Vref=1}
    \label{fig:Vref1}
\end{figure}
\begin{figure}[H]
    \centering
    \includegraphics[width=0.75\linewidth]{Secciones/Circuito4/Circuito 4 - Vref2.png}
    \caption{Ciclo de Histéresis con Vref=2}
    \label{fig:Vref2}
\end{figure}
\begin{figure}[H]
    \centering
    \includegraphics[width=0.75\linewidth]{Secciones/Circuito4/Circuito 4 - Vref3.png}
    \caption{Ciclo de Histéresis con Vref=3}
    \label{fig:Vref3}
\end{figure}

\begin{figure}[H]
    \centering
    \includegraphics[width=0.75\linewidth]{Secciones/Circuito4/Circuito 4 - Vref4.png}
    \caption{Ciclo de Histéresis con Vref=4}
    \label{fig:Vref4}
\end{figure}

Se realizaron diversas mediciones variando la tensión de referencia, según las simulaciones realizadas usando el osciloscopio en modo XY. Los resultados obtenidos fueron muy similares. El eje X representa la tensión de entrada y el eje Y la tensión de salida.

\begin{figure}[H]
    \centering
    \includegraphics[width=0.75\linewidth]{Secciones/Circuito4/SDS00004.jpg}
    \caption{Ciclo de Histéresis medido con Vref=1}
    \label{fig:enter-label}
\end{figure}

\begin{figure}[H]
    \centering
    \includegraphics[width=0.75\linewidth]{Secciones/Circuito4/SDS00005.jpg}
    \caption{Ciclo de Histéresis medido con Vref=2}
    \label{fig:enter-label}
\end{figure}

\begin{figure}[H]
    \centering
    \includegraphics[width=0.75\linewidth]{Secciones/Circuito4/SDS00006.jpg}
    \caption{Ciclo de Histéresis medido con Vref=3}
    \label{fig:enter-label}
\end{figure}

\begin{figure}[H]
    \centering
    \includegraphics[width=0.75\linewidth]{Secciones/Circuito4/SDS00007.jpg}
    \caption{Ciclo de Histéresis medido con Vref=4}
    \label{fig:enter-label}
\end{figure}